\section{Conclusions and Future Work}
\label{sec:Conclusions_and_Future_Work}

\noindent This paper describes a novel cascading verification method that uses composite reasoning over high-level system specifications and formalized domain knowledge to synthesize both system models and their desired behavioral properties. With cascading verification, model builders use a high-level DSL to encode system specifications that can be analyzed with model checking. Domain knowledge is encoded in OWL+SWRL and Prolog, which are combined to overcome their individual limitations. Synthesized DTMC models and PCTL properties are analyzed with the probabilistic model checker PRISM\@. Cascading verification was illustrated with a prototype system that verified the correctness of UAV mission plans. An evaluation of this prototype revealed non-trivial reductions in the size and complexity of input system specifications compared to the artifacts synthesized for PRISM\@.

We have identified several promising directions for future work. Composite CVC inferences are currently unidirectional, with Prolog facts derived from knowledge encoded in OWL+SWRL\@. While conceptually and practically appealing, this inference \emph{pipeline} constrains the reasoning process from refining Prolog inferences with ontological knowledge, and increases the potential for knowledge duplication. We aim to address these limitations by developing a knowledge representation framework that can support more flexible, iterative reasoning.

A second issue pertains to the artifacts that constitute the CVC knowledge base including CEMO, the Prolog rule-base, and the DTMC and PCTL templates. These artifacts should be extensible to reflect changes in domain knowledge. Extensions should in turn be verifiable to ensure that domain knowledge remains consistent across the entire knowledge base. This requirement provides impetus for the development of a mechanism that will automate the consistency management process.

Finally, we intend to further the evaluation of our method and prototype by enhancing the sophistication of the mission specification language and domain model presented in this paper.
