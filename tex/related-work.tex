\section{Related Work}
\label{sec:Related_Work}

\noindent Cascading verification can be considered a form of semantic model checking, which has been studied exclusively in the context of the Web service domain.

Narayanan and McIlraith encode Web service capability descriptions and behavioral properties with DAML-S and Petri net formalisms, respectively~\cite{Narayanan_2002}. DAML-S is a DA\-ML+OIL ontology for describing Web services. For any given Web service, an implemented system generates the Petri net corresponding to the DAML-S description of that service. The resulting net is used by KarmaSIM, a modeling and simulation environment, to perform various analysis techniques including reachability analysis and deadlock detection.

The model checking algorithm presented by Di Pietro et al.\ uses a DL-based ontology to formalize the Web service domain~\cite{Di_Pietro_2012}. The behavior of each Web service is modeled as a \emph{state transition system} (STS), while behavioral requirements are encoded with CTL\@. Both STS and CTL formalisms are extended with semantic annotations. For any given Web service, the algorithm generates a finite STS corresponding to the annotated description of that service. The resulting model is verified with model checking. The same algorithm is used by Boaro et al.\ to verify Web service security requirements~\cite{Boaro_2010}.

Oghabi et al.\ use OWL-S, an OWL ontology that supersedes DAML-S, to describe Web service behavior~\cite{Oghabi_2011}. For any given Web service, an implemented system generates a PRISM model corresponding to the OWL-S description of that service. The resulting model is verified with PRISM\@. Ankolekar et al.\ translate OWL-S process models to equivalent PROMELA models, which are verified with the SPIN model checker~\cite{Ankolekar_2005}. Liu et al.\ extend OWL-S with multiple annotation layers for specifying Web service flow properties including temporal constraints~\cite{Liu_2008}. Annotated OWL-S models are transformed to corresponding \emph{time constraint Petri net} (TCPN) models, which are verified with model checking. Lomuscio and Solanki express OWL-S process models with the \emph{interpreted systems programming language} (ISPL)~\cite{Lomuscio_2009}. ISPL is the system description language for MCMAS, a symbolic model checker tailored to the verification of multi-agent systems. In this context, Web services and Web service compositions are viewed as agents and multi-agent systems, respectively.

Our method is perhaps most compatible with the work presented by Oghabi et al. Similarities include the motivation to verify stochastic behavior and the development of a system that synthesizes PRISM models from OWL knowledge. But unlike our prototype, the system developed by Oghabi et al.\ does not synthesize behavioral properties, nor does it exploit inferred knowledge to support the synthesis of DTMC artifacts. Inferred knowledge is utilized in other work including that of Narayanan and McIlraith, Di Pietro et al.\ and Boaro et al. But existing work is exclusively concerned with the verification of Web services, and does not address the expressive and reasoning limitations that constrain OWL\@. The work presented in this paper is (to our knowledge) unique because it addresses semantic model checking limitations, and applies the resulting method to a novel domain.
